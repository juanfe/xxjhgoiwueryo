
\title{Liquidity Spot formulation}
\author{
        Juan Fernando Jaramillo \\
        IT Vichara Bogota\\
        Bogot�, \underline{Colombia}
}
\date{\today}

\documentclass[12pt]{article}

\usepackage{amsmath}
%\usepackage{footnote}

\begin{document}
\maketitle

\begin{abstract}
This paper, describe some formulation, in relation with the liquidity spot engine.
\end{abstract}

%\section{Introduction}
%This is time for all good men to come to the aid of their party!

%\paragraph{Outline}
%The remainder of this article is organized as follows.
%Section~\ref{previous work} gives account of previous work.
%Our new and exciting results are described in Section~\ref{results}.
%Finally, Section~\ref{conclusions} gives the conclusions.

\section{Previous work}\label{previous work}

This work, is based in the work by John duHadway \cite{duHadway:11doc} and \cite{duHadway:11xls}.

\section{Assets}\label{assets}

Bids are sorted in four categories, that produce assets that found the loans, this categories are
Specific/competitive, Specific/noncompetitive, General/competitive and General/noncompetitive \cite{duHadway:11doc}.
This assets with the origin of \cite{duHadway:11xls} and \cite{Jaramillo:12xls} in this order are: 

\begin{description}
	\item[$a_{ij}^{(sc)}$] Assets assigned from Specified Competitive bids; G8:L27 in \cite{duHadway:11xls}, and
		G29:L48 in \cite{Jaramillo:12-xls}.
	\item[$a_{ij}^{(sn)}$] Assets assigned from Specified Non Competitive bids; G37:L56 and G79:L98.
	\item[$a_{ij}^{(gc)}$] Assets assigned from General Competitive bids; G119:138 and G161:L180.
	\item[$a_{ij}^{(gn)}$] Assets assigned from General Non Competitive bids; G159:G178 and G201:G220.
	\item[$a_{ij}$] Assets assigned, G195:L214 and G237:L256.
\end{description}

Here the $i$ run over the the bids $B_i$, and $j$ run over the loans $L_j$.

\section{Rates}\label{rates}

There are varios rate involved in the process:

\subsection{Rate for the Mortgage Originator}

There are two Mortgage Originaror in the engine \cite{duHadway:11xls}, we will use $R_tMO^*$ for those used in
cell C219 and C261, and $R_tMO^\dagger$ used in cell M219, M261. The formula for 

\begin{gather}
	R_tMO^*=\frac{\sum_i \sum_j a_{ij}(r^*_j + LSSpread)}{\sum_i \sum_j a_{ij}} 
	= \frac{\sum_i \sum_j a_{ij}r^*_j}{\sum_i \sum_j a_{ij}} + LSSpread
\end{gather}

And

\begin{gather}
	R_tMO^\dagger = R_tSSI + LSSpread = \frac{\sum_i \sum_j a_{ij}r^\dagger _i}{\sum_i \sum_j a_{ij}} + LSSpread
\end{gather}

The difference are in the rates, $r^*_j$ are rate per loan $j$, and $r^\dagger _i$ are rate per bid $i$.

\subsubsection{$r^*_j$ rate per loan $j$}

$r^*_j$ are in G187:L187 and G229:L229. 

\subsubsection{$r^\dagger _i$ rate per bid $i$}

$r^\dagger _i$ are in C195:C214 and C237:C256


%\section{Results}\label{results}
%In this section we describe the results.

%\section{Conclusions}\label{conclusions}
%We worked hard, and achieved very little. :-)

%\bibliographystyle{abbrv}
\bibliographystyle{plain}
\bibliography{engine_formulas}

\end{document}
