
\title{Liquidity Spot formulation}
\author{
        Juan Fernando Jaramillo \\
        IT Vichara Bogota\\
        Bogot�, \underline{Colombia}
}
\date{\today}

\documentclass[12pt]{article}

%Final rate per loan
\newcommand{\rl}{{r^{_l}}_j} 
%Final rate per bid 
\newcommand{\rb}{{r^{_b}}_i} 
%Assets assigned from Specified Competitive bids
\newcommand{\asc}{a_{ij}^{(sc)}}
%Assets assigned from Specified Non Competitive bids
\newcommand{\asn}{a_{ij}^{(sn)}}
%Assets assigned from General Competitive bids
\newcommand{\agc}{a_{ij}^{(gc)}}
%Assets assigned from General Non Competitive bids
\newcommand{\agn}{a_{ij}^{(gn)}}
%Loan Amount
\newcommand{\Loan}{L_i} %{\math{L}_j}
%Bids
\newcommand{\Bid}{B_i} %\math{B}_i}
%Bid Rate if competitive
\newcommand{\Rbc}{R^{(bc)}_i}
%Loan Rate
\newcommand{\Rl}{R^{(l)}}
%Rate Awarded 
\newcommand{\Raw}{R^{(aw)}_i}
%Working Average Rate for Specified Bids
\newcommand{\Rwas}{R^{(was)}}
%Working Average Rate for Specified Competitive Bids
\newcommand{\Rwasc}{R^{(wasc)}}
%Working Average Rate for Specified Non Competitive Bids
\newcommand{\Rwasn}{R^{(wasn)}}
%Working Avergage Rate for General Competitive Bids
\newcommand{\Rwagc}{R^{(wagc)}}
%Working Avergage Rate for General Non Competitive Bids
\newcommand{\Rwagn}{R^{(wagn)}}
\newcommand{\rgcl}{r^{(gcl)}}
\newcommand{\rwagcl}{r^{(wagcl)}}
%Market Rate differential
\newcommand{\MRdif}{M^{(r})}
%Subscription
\newcommand{\Subs}{P^{(subs)}_j}
%Highest daily rate
\newcommand{\HiDayRt}{R^{(drt)}}

\usepackage{amsmath}
\begin{document}
\maketitle

\begin{abstract}
This paper, describe some formulation, in relation with the liquidity spot engine.
\end{abstract}

%\section{Introduction}
%This is time for all good men to come to the aid of their party!

%\paragraph{Outline}
%The remainder of this article is organized as follows.
%Section~\ref{previous work} gives account of previous work.
%Our new and exciting results are described in Section~\ref{results}.
%Finally, Section~\ref{conclusions} gives the conclusions.

\section{Previous work}\label{previous work}

This work, is based in the work by John duHadway \cite{ddoc} and \cite{dx}.

\section{Assets}\label{assets}

Bids are sorted in four categories, that produce assets that found the loans, this categories are
Specific/competitive, Specific/noncompetitive, General/competitive and General/noncompetitive \cite{ddoc}.
This assets with the origin of \cite{dx} and \cite{Jx} in this order are: 

\begin{description}
	\item[$\asc$] Assets assigned from Specified Competitive bids; G8:L27 in \cite{dx}, and
		G29:L48 in \cite{Jx}.
	\item[$\asn$] Assets assigned from Specified Non Competitive bids; G37:L56 and G79:L98.
	\item[$\agc$] Assets assigned from General Competitive bids; G119:138 and G161:L180.
	\item[$\agn$] Assets assigned from General Non Competitive bids; G159:G178 and G201:G220.
	\item[$a_{ij}$] Assets assigned, G195:L214 and G237:L256.
\end{description}

Here the $i$ run over the the bids $\Bid$, and $j$ run over the loans $\Loan$.

\section{Rates}\label{rates}

There are varios rate involved in the process:

\subsection{Rate for the Mortgage Originator}

There are two Mortgage Originaror in the engine \cite{dx}, we will use $RMO_b$ for those used in
cell C219 and C261, and $RtMO_l$ used in cell M219, M261. The formula are:

\begin{gather}
	RtMO_l=\frac{\sum_i \sum_j a_{ij}(\rl + LSSpread)}{\sum_i \sum_j a_{ij}} 
		= \frac{\sum_i \sum_j a_{ij}\rl}{\sum_i \sum_j a_{ij}} + LSSpread
\end{gather}

And

\begin{gather}
	RtMO_b = RtSSI + LSSpread = \frac{\sum_i \sum_j a_{ij}\rb}{\sum_i \sum_j a_{ij}} + LSSpread
\end{gather}

The difference are in the rates, $\rl$ are rate per loan $j$, and $\rb$ are rate per bid $i$.

\subsubsection{$\rl$ rate per loan $j$}

$\rl$ are in G187:L187\cite{dx} and G229:L229\cite{Jx}. This is calculate using the next formula:

\begin{gather}
	\rl = \frac{\sum_j RL_j \asn}{\sum_j \asn} + \Raw + \max {(\Rwas , \Rwagc)}
\end{gather}

Where $\Raw$ is the Rate Awarded by Specified bid, E119:E137\cite{dx} and E161:E180\cite{Jx}. And $\Rwas$ is the
working average rate for the specified bids M67\cite{dx} and M109\cite{Jx}, and $\Rwagc$ are the working average
rate for the general competitive bids, M146\cite{dx} and M190\cite{Jx}.  

\begin{gather}\label{Raw}
	\Raw = \Rbc + \frac{ \sum _j \agc (\rgcl - \rwagcl}{\sum _j \agc}
\end{gather}

Where $\rgcl$  G146:M146\cite{dx} and G188:M188\cite{Jx} is:

\begin{gather}\label{rgcl}
	\rgcl = \rwagcl + (1-\Subs)*\MRdif
\end{gather}

Where $\MRdif$ are the Market Rate differential D144\cite{dx} and D186\cite{Jx}, and $\Subs$ are the
subscription percentage per loan G66:L66\cite{dx} and G108:L108\cite{Jx}:

\begin{gather}
	\MRdif = \HiDayRt - D185
\end{gather}

Where $\HiDayRt$ is the highest daily rate D142\cite{dx} and D184\cite{Jx}.

\begin{gather}
	\HiDayRt = \max (\max_i (\Rbc), \max_j \Rl)) 
\end{gather}

Where $\Rl$ is the rate per loan $j$, calculate in the step 3. G36:L36 and G57:L57\cite{Jx}.

\begin{gather}
	\Subs = \frac{\sum _i (\asc + \asn)}{\Loan}
\end{gather}

And $\rwagcl$ G140:M140\cite{dx} and G182:M182\cite{Jx}, as you can see it in \ref{rgcl} is not used,
because as you replace in \ref{Raw} it dissapear.

\begin{gather}
	\Rwas = \frac{(\Rwasc * \sum _i \sum _j \asc) + (\Rwasn * \sum _i \sum _j \asn)}{\sum _i \sum _j (\asc + \asn)}
\end{gather}

Where $\Rwasc$, $\Rwasn$ are the working average rates for specified competitive and non competite bids, M30,
M59\cite{dx} and M51, M101\cite{Jx}.

\subsubsection{$\rb$ rate per bid $i$}

$\rb$ are in C195:C214\cite{dx} and C237:C256\cite{Jx}.

%\section{Results}\label{results}
%In this section we describe the results.

%\section{Conclusions}\label{conclusions}
%We worked hard, and achieved very little. :-)

%\bibliographystyle{abbrv}
\bibliographystyle{plain}
\bibliography{engine_formulas}

\end{document}
